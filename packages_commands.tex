\documentclass[11pt,a4paper]{article}
\usepackage[utf8]{inputenc}
\usepackage[T1]{fontenc}
\usepackage[spanish]{babel}
\usepackage{amsmath}
\usepackage{amsfonts}
\usepackage{amssymb}
\usepackage{makeidx}
\usepackage{graphicx}
\usepackage{lmodern}
\usepackage{kpfonts}
\usepackage[left=2cm,right=2cm,top=2cm,bottom=2cm]{geometry}

\usepackage{cprotect}
\usepackage{fancyhdr}
\pagestyle{fancy}
\fancyhf{}
\rhead{Álvaro Hurtado}
\lhead{DGME}
\rfoot{\thepage}

\author{Álvaro Hurtado}
\title{Paquetes y comandos más utilizados}
\date{}

\begin{document}
\maketitle
\tableofcontents

\section{Paquetes}

\begin{enumerate}

\item \textbf{Da color al documento:}
\begin{verbatim}
\usepackage{xcolor}
\end{verbatim}

\item \textbf{Reduce los márgenes laterales para que la parte donde se escribe abarque más en cada página:}
\begin{verbatim}
\usepackage{fullpage}
\end{verbatim}

\item \textbf{Permite dibujar gráficas:}
\begin{verbatim}
\usepackage{tkz-fct}
\end{verbatim}

\item \textbf{Permite dibujar figuras:}
\begin{verbatim}
\usepackage{tikz}
\end{verbatim}

\item \textbf{Permite utilizar Maxima dentro del documento e imprimir la solución a los problemas sin necesidad de intermediarios o copiar y pegar partes de texto:}
\begin{verbatim}
\usepackage{maxiplot}
\end{verbatim}

\item \textbf{Permite modificar la tabla de contenidos (índice de contenidos) del documento:}
\begin{verbatim}
\usepackage{tocloft}
\renewcommand{\cftsecdotsep}{.} %Pone puntos entre el "leader" y la página
\end{verbatim}

\item \textbf{Permite introducir hipervínculos en nuestro documento:}
\begin{verbatim}
\usepackage[hidelinks]{hyperref} %hidelinks para que no aparezcan las cajas
\end{verbatim}
(el comando \verb|\url{direccion}| muestra la dirección y el comando\\ \verb|\href{direccion}{texto alternativo}| muestra un texto diferente)

\item \textbf{Permite incluir teoremas, demostraciones, etc; en tu documento:}
\begin{verbatim}
\usepackage{amsthm}

\theoremstyle{definition}
\newtheorem{definition}{Definición}[chapter]
\newtheorem{example}{Ejemplo}[section]
\theoremstyle{plain}
\newtheorem{theorem}{Teorema}[chapter]
\newtheorem{proposition}[theorem]{Proposición}
\newtheorem{lemma}[theorem]{Lema}
\newtheorem{corollary}[theorem]{Corolario}
\theoremstyle{remark}
\newtheorem{remark}{Observación}[chapter]

\renewcommand\qedsymbol{$\blacksquare$}
\end{verbatim}
(Para un artículo cambiar \verb|chapter| por \verb|section|)

\item \textbf{Permite crear subfiguras dentro del entorno \texttt{figure}:}
\begin{verbatim}
\usepackage{subfigure}
\end{verbatim}
(Para utilizarlo es necesario abrir dentro del entorno \texttt{figure} otro entorno llamado \texttt{subfigure})

\item \cprotect\textbf{Permite hacer referencias de una manera más cómoda, eliminando la obligación de añadir prefijos a \verb|\ref{•}|:}
\begin{verbatim}
\usepackage{cleveref}
\end{verbatim}
(define el comando \verb|\cref{•}|)

\item \textbf{Permite añadir comentarios (p.ej. para numerarlas) a las filas y columnas de una matriz:}
\begin{verbatim}
\usepackage{kbordermatrix}
\end{verbatim}

\item \textbf{Permite imprimir texto de manera literal, es decir, \LaTeX~ ignora el contenido:} \textit{(ver siguiente apartado)}
\begin{verbatim}
\usepackage{verbatim}
\end{verbatim}
(Mejora el entorno \verb|verbatim| que ya contiene \LaTeX~ por defecto, haciendo posible un mayor control de los espacios dentro de dicho entorno y añadiendo el entono \verb|comment|, con el mismo efecto de \verb|verbatim| pero sin imprimir el contenido)

\item \cprotect\textbf{Permite utilizar \verb|\verb| en diferentes entornos:}
\begin{verbatim}
\usepackage{cprotect}
\end{verbatim}
(para llevar a cabo esta acción es necesario incluir \verb|\cprotect| antes del entorno donde se desea incluir dicho comando)

\item \textbf{Permite tener un mayor control de los encabezados y pies de página:}
\begin{verbatim}
\usepackage{fancyhdr}
\pagestyle{fancy}
\fancyhf{}
\rhead{Álvaro Hurtado}
\lhead{DGME}
\rfoot{\thepage}
\end{verbatim}

\item \cprotect \textbf{Permiten usar el comando \verb|\mathscr{•}|, que nos proporciona una manera de escribir letras elegantes:}
\begin{enumerate}
	\item \verb|\usepackage[mathscr]{euscript}|
	\item \verb|\usepackage{mathrsfs}|
\end{enumerate}

\item \textbf{Añade una estética diferente a los títulos de los capítulos y permite escribir citas junto a estos de una manera más agradable a la vista:}
\begin{verbatim}
	\usepackage{quotchap}
\end{verbatim}

\item \cprotect \textbf{Permite escribir letras minúsculas en el estilo de letra de \verb|\mathbb{•}|:}
\begin{verbatim}
	\usepackage{bbm}
\end{verbatim}
(para poder llevar acabo dicha acción es necesario escribir el comando \verb|\mathbbm{•}|)

\item \textbf{Te permite tachar con una línea oblicua dentro del modo matemático:}
\begin{verbatim}
	\usepackage{cancel}
\end{verbatim}
Añade los siguientes comandos:
\begin{verbatim}
	\cancel{text} #Tacha el texto con una barra
	\cancelto{text1}{text2} #Tacha el texto con una flecha y agrega por que se sustituye
	\xcancel{text} #Tacha con una X
	\bcancel{text} #Tacha con una contrabarra
\end{verbatim}

\end{enumerate}

\section{Comandos}

\begin{enumerate}

\item \verb|\newcommand{\R}{\mathbb{R}}|

\item \verb|\newcommand{\then}{~\Longrightarrow~}|

\item \verb|\newcommand{\tq}{\mbox{ t.q. }}|

\item Comando para derivadas parciales:
\begin{verbatim}
\newcommand{\Partial}[2]{\frac{\partial #1}{\partial #2}}
\end{verbatim}

\item Comando para derivadas en Notación de Leibniz:
\begin{verbatim}
\newcommand{\D}[2]{\frac{d #1}{d #2}}
\end{verbatim}

\item Añadir en la primera línea de la bibliografía para que esta aparezca en el TOC:
	\begin{enumerate}
		\item Para artículos: \verb|\addcontentsline{toc}{section}{\refname}|
		\item Para libros: \verb|\addcontentsline{toc}{chapter}{\refname}|
	\end{enumerate}
	(En general, para añadir un apartado nuevo al TOC, utilizaremos el comando \linebreak \verb|\addcontentsline{text}{secunit}{entry}|. \texttt{text} es donde queremos añadir el apartado; \texttt{secunit} es el tipo de entrada que queremos que sea, por ejemplo, \texttt{chapter}, \texttt{section}, etc; y \texttt{entry} es la entrada que queremos que aparezca en \texttt{text})

\item Cambiar el nombre de la bibliografía (antes del entorno):\\
\verb|\renewcommand{\refname}{Bibliografía}|

\item \verb|\newcommand{\latex}{\LaTeX~}|

\item Clausura de un conjunto (closure):
\begin{verbatim}
	\newcommand{\clo}[1]{\overline{#1}}
\end{verbatim}
o bien
\begin{verbatim}
	\newcommand{\clo}[1]{\bar{#1}}
\end{verbatim}

\item Conjunto complementario:
\begin{verbatim}
	\newcommand{\comp}[1]{#1^\complement}
\end{verbatim}

\end{enumerate}

\section{Entornos}

\begin{enumerate}
	
\item Recuadro sencillo para insertar texto:
\begin{verbatim}
	\newenvironment{Boxed}{
		\begin{center}
			\begin{tabular}{|p{0.9\textwidth}|}
				\hline \vspace{0.5mm}}{
				\vspace{2mm} \\ \hline
			\end{tabular}
	\end{center}}
\end{verbatim}

\item Define un entorno \texttt{note} para introducir notas entre dos líneas horizontales:
\begin{verbatim}
	\newenvironment{note}{
		\begin{center}
			\begin{tabular}{p{0.9\textwidth}}
				\hline \\
				\textbf{NOTA: }}{
				\\ \\ \hline
			\end{tabular}
	\end{center}}
\end{verbatim}

\end{enumerate}

\end{document}